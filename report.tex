\documentclass[a4paper,12pt,oneside]{article}
\usepackage{amsmath}
\usepackage{mathtools}
% \usepackage{caption}
\usepackage[labelformat=empty]{caption}
\usepackage{mathptmx}
\usepackage{fixltx2e}
\usepackage{graphicx}
\usepackage[margin=1.0in]{geometry}
\usepackage{float}
\let\counterwithout\relax
\let\counterwithin\relax
\usepackage{setspace}
\usepackage{chngcntr}
\usepackage{fancyhdr}
\usepackage{etoolbox}
\patchcmd{\thebibliography}{\section*{\refname}}{}{}{}


\pagestyle{fancy}
\fancyhf{}
\rfoot{\thepage}
%\renewcommand{\headrulewidth}{0.0pt}
%\renewcommand{\footrulewidth}{0.0pt}


\begin{document}
\thispagestyle{empty}
%\pagenumbering{gobble}
\begin{center}

\large{\textbf{{Generating Usability Reports from User Inputs and Eye Movements}}}
\setlength{\baselineskip}{1.5\baselineskip}
\\
\vspace{5mm}
\textbf{PROJECT REPORT}

Submitted in the partial fulfilment of the award of the degree
of
\\
\textbf{Bachelor of Technology}
\\
in
\\
\textbf{Computer Science \& Engineering}
\\
of
\\
\textbf{APJ Abdul Kalam Technological University}
\\
by
\\
\textbf{Ganesh Sekhar}
\\
\textbf{Sachin Sajan Punnoose}
\\
\textbf{Shan Eapen Koshy}
\\
\textbf{S Hemanth}
\\
\vspace{5mm}
\begin{figure}[H]
\centering
\includegraphics[width=4cm]{ceclogo.png}
\end{figure}
\textbf{November 2019}
\vspace{8mm}
\\
Department of Computer Engineering
\\
College of Engineering, Chengannur, Kerala -689121
\\
Phone: (0479) 2454125, 2451424; Fax: (0479) 2451424
\\
\end{center}

\newpage
\thispagestyle{empty}
\begin{center}
\setlength{\baselineskip}{1.5\baselineskip}
{\large\textbf{COLLEGE OF ENGINEERING, CHENGANNUR}}
\\
{\large\textbf{KERALA}}
\\
\begin{figure}[H]
\centering
\includegraphics[width=4cm]{ceclogo.png}
\end{figure}
\setlength{\baselineskip}{1.5\baselineskip}
\textbf{Department of Computer Engineering}
\\
\textbf{CERTIFICATE}
\\
This is to certify that the seminar entitled
\\
\textbf{Generating Usability Reports from User Inputs and Eye Movements}
\\
Submitted by
\\
\textbf{Ganesh Sekhar}
\\
\textbf{Sachin Sajan Punnoose}
\\
\textbf{Shan Eapen Koshy}
\\
\textbf{S Hemanth}
\\
is a bonafide record of the work done by him.
\end{center}
\vspace{20ex}
\hspace{5ex}
% \textbf{Mrs.Shiny B}
\hspace{15ex}
% \textbf{Mrs.Shiny B}
\hspace{16ex}
% \textbf{Dr. Smitha Dharan}
\\
\vspace{0ex}
\hspace{4ex}
\textbf{Co-ordinator}
\hspace{20ex}
\textbf{Guide}
\hspace{16ex}
\textbf{Head of the Department} 
\newpage
\pagenumbering{roman}
\renewcommand{\headrulewidth}{0.0pt}
\renewcommand{\footrulewidth}{0.0pt}
\begin{center}
\large{\textbf{ACKNOWLEDGEMENT}}
\end{center}
\vspace{6ex}
\setlength{\baselineskip}{1.5\baselineskip}
\paragraph{}
I am greatly indebted to \textbf{God Almighty} for being the guiding light throughout with his
abundant grace and blessings that strengthened me to do this endeavour with confidence.
\paragraph{}
I express my heartfelt gratitude towards \textbf{Dr. Jacob Thomas V.}, Principal, College
of Engineering Chengannur for extending all the facilities required for doing my seminar.
I would also like to thank \textbf{Dr. Smitha Dharan}, Head, Department of Computer
Engineering, for providing constant support and encouragement.
\paragraph{}
Now I extend my sincere thanks to my seminar co-ordinators \textbf{Mrs. Shiny B}, Assistant
Professor in Computer Engineering for guiding me in my work and providing timely
advices and valuable suggestions.
\paragraph{}
Last but not the least, I extend my heartfelt gratitude to my parents and friends for
their support and assistance.	
\pagenumbering{gobble}

\newpage
\begin{center}
\large{\textbf{ABSTRACT}}
\end{center}
\vspace{4ex}
\paragraph{}
Usability testing is a technique used to evaluate a product by testing it on users. It is an important factor in marketing a product since it gives a complete structure of how the users use the product.

After understanding how real users interact with your product, you can improve the product based on the results. The primary purpose of a usability test is to improve it’s designed so as to make it more user-friendly.

The proposed system uses eye detection to locate the positions on the screen where the user pays more attention and a heat map is generated from it. This testing is done for different age groups and a final report listing all the findings (positives and negatives) is generated. Positive findings will help the team to know that they’re on the right track and the negative findings provide proposals to solve them



\setlength{\baselineskip}{1.0\baselineskip}

% Table of content Page
\setlength{\baselineskip}{1.0\baselineskip}
\newpage
\begin{center}
\tableofcontents
\end{center}

% List of Figures Page
\newpage
\thispagestyle{plain}
\begin{center}
\listoffigures
\end{center}

% List of Tables Page
\newpage
\thispagestyle{plain}
\begin{center}
\listoftables
\end{center}

\newpage
\rfoot{\thepage}
\lhead{\textit{Generating Usability Reports from User Inputs and Eye Movements}}
\lfoot{\textit{College of Engineering Chengannur}}
\rfoot{\thepage}
\renewcommand{\headrulewidth}{0.0pt}
\renewcommand{\footrulewidth}{0.0pt}
\renewcommand{\headrulewidth}{0.0pt}
\renewcommand{\footrulewidth}{0.0pt}
\section{INTRODUCTION}
\pagenumbering{arabic}
\paragraph{}


\newpage
\section{PROBLEM FORMULATION}
\paragraph{}

\newpage
\section{LITERATURE SURVEY}
\subsection{Eye-Tracking}
    \begin{itemize} 
        \item \textbf{PACE}, a Personalized, Automatically Calibrating Eye-tracking system that identifies and collects data unobtrusively from user interaction events on standard computing systems without the need for specialized equipment. PACE relies on eye/facial analysis of webcam data based on a set of robust geometric gaze features and a two-layer data validation mechanism to identify good training samples from daily interaction data. The design of the system is founded on an in-depth investigation of the relationship between gaze patterns and interaction cues, and takes into consideration user preferences and habits. The result is an adaptive, data-driven approach that continuously recalibrates, adapts and improves with additional use. Quantitative evaluation on 31 subjects across different interaction behaviors shows that training instances identified by the PACE data collection have higher gaze point-interaction cue consistency than those identified by conventional approaches. An in-situ study using real-life tasks on a diverse set of interactive applications demonstrates that the PACE gaze estimation achieves an average error of 2.56º , which is comparable to state-of-the-art, but without the need for explicit training or calibration.
        \item \textbf{TurkerGaze}, 
        Traditional eye tracking requires specialized hardware, which means collecting gaze data from many observers is expensive, tedious and slow. Therefore, existing saliency prediction datasets are order-of-magnitudes smaller than typical datasets for other vision recognition tasks. The small size of these datasets limits the potential for training data intensive algorithms, and causes overfitting in benchmark evaluation. To address this deficiency, this paper introduces a webcam-based gaze tracking system that supports large-scale, crowdsourced eye tracking deployed on Amazon Mechanical Turk (AMTurk). By a combination of careful algorithm and gaming protocol design, our system obtains eye tracking data for saliency prediction comparable to data gathered in a traditional lab setting, with relatively lower cost and less effort on the part of the researchers. Using this tool, we build a saliency dataset for a large number of natural images. We will open-source our tool and provide a web server where researchers can upload their images to get eye tracking results from AMTurk.
        \item WebGazer
    \end{itemize}

\subsection{Usability Testing}

\newpage
\section{RELATED WORKS}
\subsection{Tobii}
\paragraph{}
Web-based Usability testing tool for quick and easy user testing of web-sites or digital products. Live viewing of where the user is looking and generates a timeline view of eye tracked.

\subsection{Nielsen Norman Research Study}
\paragraph{}
The Nielsen Norman Group is an American computer user  interface and user experience consulting firm.
\subsection{usertesting.com}

\newpage
\section{PROPOSED SYSTEM}
\paragraph{}
In this proposed system, a user can submit a URL of the website to be analyzed. The system then generates a unique URL for this experiment which can be manually shared to different users.
Testers can access this URL and interact with the website normally while we collect the tester's eye coordinates that we obtained through webgazer.js. Basic demographic of the tester such as age and gender are also collected for categorization and report generation. The collected data is then stored in the server. 
The testing details can be reviewed from the admin's dashboard. Several features such as timeline, demographic filtering, heatmap, etc, are provided for easily analyzing the data.


\newpage
\section{SYSTEM DESIGN}
    \begin{figure}[H]
    \includegraphics[scale=0.5]{DFD.png}
    %\counterwithin{figure}{section}
    \centering
    \caption[ Data Flow Diagrams]{\textbf{Figure 1.}  Data Flow Diagrams}
    \end{figure}




\newpage
\section{CONCLUSION}
\paragraph{}


\newpage
% \cleardoublepage
% \addcontentsline{toc}{section}{\textbf{References}}
% \addcontentsline{}{section}{\textbf{References}}

s\section{REFERENCES}
\begin{thebibliography}{9}
    \bibitem{d}
    Papoutsaki, Alexandra \& Sangkloy, Patsorn \& Laskey, James \& Daskalova, Nediyana \& Huang, Jeff \& Hays, James. (2016). \emph{WebGazer: Scalable Webcam Eye Tracking Using User Interactions.}

    \bibitem{d}
    Eyetribe.com
    \bibitem{d}
    Sticky by Tobii Pro
    \bibitem{d}
    Kiril Alexiev, Teodor Toshkov and Peter Dojnow. 2019. Accuracy and Precision of eye tracker by head movement compensation and calibration. \emph{20th International Conference on Computer Systems and Technologies
(CompSysTech'19)}, Jun 21-22, 2019, Ruse, Bulgaria, 8 pages.
https://doi.org/10.1145/3345252.3345278.



\end{thebibliography}
\end{document}